\documentclass[10pt, a4paper]{article}\usepackage[]{graphicx}\usepackage[]{color}
% maxwidth is the original width if it is less than linewidth
% otherwise use linewidth (to make sure the graphics do not exceed the margin)
\makeatletter
\def\maxwidth{ %
  \ifdim\Gin@nat@width>\linewidth
    \linewidth
  \else
    \Gin@nat@width
  \fi
}
\makeatother

\definecolor{fgcolor}{rgb}{0.345, 0.345, 0.345}
\newcommand{\hlnum}[1]{\textcolor[rgb]{0.686,0.059,0.569}{#1}}%
\newcommand{\hlstr}[1]{\textcolor[rgb]{0.192,0.494,0.8}{#1}}%
\newcommand{\hlcom}[1]{\textcolor[rgb]{0.678,0.584,0.686}{\textit{#1}}}%
\newcommand{\hlopt}[1]{\textcolor[rgb]{0,0,0}{#1}}%
\newcommand{\hlstd}[1]{\textcolor[rgb]{0.345,0.345,0.345}{#1}}%
\newcommand{\hlkwa}[1]{\textcolor[rgb]{0.161,0.373,0.58}{\textbf{#1}}}%
\newcommand{\hlkwb}[1]{\textcolor[rgb]{0.69,0.353,0.396}{#1}}%
\newcommand{\hlkwc}[1]{\textcolor[rgb]{0.333,0.667,0.333}{#1}}%
\newcommand{\hlkwd}[1]{\textcolor[rgb]{0.737,0.353,0.396}{\textbf{#1}}}%
\let\hlipl\hlkwb

\usepackage{framed}
\makeatletter
\newenvironment{kframe}{%
 \def\at@end@of@kframe{}%
 \ifinner\ifhmode%
  \def\at@end@of@kframe{\end{minipage}}%
  \begin{minipage}{\columnwidth}%
 \fi\fi%
 \def\FrameCommand##1{\hskip\@totalleftmargin \hskip-\fboxsep
 \colorbox{shadecolor}{##1}\hskip-\fboxsep
     % There is no \\@totalrightmargin, so:
     \hskip-\linewidth \hskip-\@totalleftmargin \hskip\columnwidth}%
 \MakeFramed {\advance\hsize-\width
   \@totalleftmargin\z@ \linewidth\hsize
   \@setminipage}}%
 {\par\unskip\endMakeFramed%
 \at@end@of@kframe}
\makeatother

\definecolor{shadecolor}{rgb}{.97, .97, .97}
\definecolor{messagecolor}{rgb}{0, 0, 0}
\definecolor{warningcolor}{rgb}{1, 0, 1}
\definecolor{errorcolor}{rgb}{1, 0, 0}
\newenvironment{knitrout}{}{} % an empty environment to be redefined in TeX

\usepackage{alltt}
    \pagestyle{empty}
    \renewcommand{\baselinestretch}{1.5}
    \usepackage[english]{babel}
 
\usepackage[top=2.2cm,bottom=2.2cm, left = 2.1cm, right = 2.1cm]{geometry}
\usepackage{latexsym}
\usepackage{graphicx}
\usepackage{amsmath, amssymb}
\usepackage[utf8]{inputenc}
\usepackage{mathpazo}
\usepackage{geometry}
\usepackage{amsfonts}
\usepackage{ifsym}
\usepackage{booktabs}
\usepackage{enumerate}
\usepackage{a4wide}
\usepackage{float}
\usepackage{dsfont}
\usepackage{hhline}
\usepackage{lscape}
\usepackage{stmaryrd}
\usepackage{paralist}
%\usepackage[utf8]{inputenc}
\usepackage{fancyhdr}
\usepackage{nicefrac}


\usepackage[usenames,x11names]{xcolor} % Die Optionen definieren zusätzliche Farben (siehe Dokumentation)
\usepackage{graphicx}
\usepackage{subfigure}
% Tikz zeug
\usepackage{tikz}
\usepackage{tkz-tab}
\newcommand{\widebar}{\overline}


%\input{definitionen}

\newfont{\suet}{suet14}
\DeclareTextFontCommand{\textsuet}{\suet}



\newcommand{\Var}{\text{Var}}
\newcommand{\Ew}{\mathbb{E}}
\newcommand{\med}{\text{med}}
\newcommand{\MSE}{\text{MSE}}
\newcommand{\Bias}{\text{Bias}}
\newcommand{\A}{\mathcal{A}}
\renewcommand{\O}{\Omega}
\newcommand{\R}{\mathbb{R}}
\newcommand{\N}{\mathbb{N}}
\newcommand{\Borell}{\mathfrak{B}}
\renewcommand{\P}{\mathcal{P}}
\newcommand{\vt}{\vartheta}
\renewcommand{\le}{\leqslant} % ich finde Kleinergleich mit schrägen Strich schöner
\renewcommand{\ge}{\geqslant}

%-- charakteristische-Funktion-/Indikatorfunktion-Eins '\ind'
\usepackage{silence}
\WarningFilter{latexfont}{Size substitutions with differences}
\WarningFilter{latexfont}{Font shape `U/bbold/m/n' in size}
\DeclareSymbolFont{bbold}{U}{bbold}{m}{n}
\DeclareSymbolFontAlphabet{\mathbbold}{bbold}
\newcommand{\ind}{\mathbbold{1}} 

    \setlength{\textwidth}{15cm}
\setlength{\oddsidemargin}{0.5cm}

\setlength{\parskip}{2ex plus0.5ex minus0.5ex}
\setlength{\parindent}{0em}
\renewcommand{\labelenumi}{\alph{enumi})}
\renewcommand{\vt}{\vartheta}
\newcommand{\E}{\mathbb{E}}
\newcommand{\V}{\mathbb{V}\text{ar}}
\newcommand{\C}{\mathbb{C}\text{ov}}
\renewcommand{\P}{\mathbb{P}}
\IfFileExists{upquote.sty}{\usepackage{upquote}}{}
\begin{document}
	
	
	
\subsection*{Simulation Study}
Let $(J_i)_{n \in \mathbb{N}}$ be a strictly stationary sequence with marginal distribution $F_J$ and right endpoint $x_R$ and an i.i.d. sequence of heavy-tailed random variables $(W_i)_{i \in \mathbb{N}}$ the waiting times between the observations $J_i$ which are independent of $J_i$ for $i \in \mathbb{N}$. Moreover, we assume that the $W_i$'s are in the domain of attraction of a strictly stable random variable $D$ with laplace transform $\mathbb{E}(e^{-sD})=\exp(-s^{\beta})$, $0<\beta<1$, i.e., there exists a regularly varying function $b \in RV(-1/\beta)$ such that
\begin{align} \label{eq4}
b(n)(W_1+\dots+W_n) \overset{d}{\longrightarrow} D.
\end{align}
Given a threshold $u \in (x_L,x_R)$, consider the stopping time 
\begin{align}
	\tau(u):=\min\{k \mid J_k > u \}
\end{align}
and define 
\begin{align}
	T(u)=\sum_{k=1}^{\tau(u)} W_k
\end{align}
the waiting time until the sequence $(J_i)$ exceeds the threshold $u$ for the first time. We will consider $T(u)$ as an estimator for the inter-arrival time of exceedances of $u$.
Under some assumptions Hees \& Fried (2019) found out that 
\begin{align}
	\frac{T(u)}{b(1/\overline{F}_J(u))} \overset{d}{\longrightarrow} Z_{\beta,\theta} \quad \text{as} \quad u \to x_R
\end{align}
with $\widebar{F}_J:=1-F_J$ and where $Z_{\beta,\theta}$ follows a mixture of the point measure in 0 and a Mittag-Leffler distribution, more precisely
\begin{align}
	Z_{\beta,\theta} \sim (1-\theta)\varepsilon_0+\theta\mathbb{P}_{\beta,\theta}
\end{align}
with $\mathbb{P}_{\beta,\theta}=ML(\beta,\theta^{-1/\beta})$.
Therefore, we assume that for $u$ high enough $T(u)$ is approximately distributed like $X_{\beta,\theta,u}:=b(1/\widebar{F}_J(u)) \cdot Z_{\beta,\theta} \sim (1-\theta)\cdot\varepsilon_0+\theta \cdot ML(\beta,b(1/\widebar{F}_J(u)) \cdot \theta^{-1/\beta})$.
\\
Since the Mittag-Leffler distribution is heavy-tailed such that there are no finite moments, we look at the fractional moments of $X_{\beta,\theta,u}$. The $q$-th fractional moment of $X$ with $q \in (0,1)$, $q<\beta$, is given by
\begin{align} 
m_q:=m_q(X_{\beta,\theta,u}):&=\mathbb{E}(X_{\beta,\theta,u}^q) = \theta \cdot \frac{ q \cdot \pi \cdot (\theta^{-1/\beta}\cdot b(1/\widebar{F}_J(u)))^q }{\beta \cdot \Gamma(1-q)\sin(\frac{q \cdot \pi}{\beta})} 
\end{align}
Hence, by calculating
\begin{align} \label{eq1}
	\frac{m_q^2}{m_{2q}}=
	\theta \cdot \frac{\frac{q\pi}{\beta}\cdot\Gamma(1-2q)\sin(\frac{2q\pi}{\beta})}{2\cdot\Gamma(1-q)^2\sin(\frac{q\pi}{\beta})^2}, \qquad \text{for \ } 2q<\beta
\end{align}
we get rid of the unknown regularly varying function $b$ and the survival propability $\widebar{F}_J(u)$.
Since $\lim\limits_{x \to 0}\frac{\sin(x\pi)}{x\pi} = 1$ (easily shown by using L'Hopital's rule) and $\Gamma(1)=1$ it follows that
\begin{align} \label{eq2}
\lim\limits_{q \to 0} \frac{m_q^2}{m_{2q}} = \theta.
\end{align}
Equations (\ref{eq1}) and (\ref{eq2}) lead to two ways of estimating $\beta$ and $\theta$ without the need of knowing $b \in RV(-1/\beta)$ or $\widebar{F}_J(u)$.
By using the empirical fractional moments
\begin{align}
	\widehat{m}_q=\frac{1}{k_u}\sum_{i=1}^{k_u} T_i^q(u)
\end{align}
where $T_i(u)$, $i=1,\dots,k_u$ are the inter-arrival times of the threshold $u$ exceedances and $k_u$ is the number of exceedances, we get the following estimator for $\theta$ depending on $\beta$:
\begin{align} \label{eq3}
	\widehat{\theta}_q(\beta)=\frac{\widehat{m}_q^2}{\widehat{m}_{2q}}\cdot \frac{2\cdot\Gamma(1-q)^2\sin(\frac{q\pi}{\beta})^2}{\frac{q\pi}{\beta}\cdot\Gamma(1-2q)\sin(\frac{2q\pi}{\beta})}.
\end{align}
Hence, we have to estimate $\beta$ first by calculating the root of $\widehat{\theta}_{q_1}(\beta)-\widehat{\theta}_{q_2}(\beta)$ on $(2\cdot\max(q_1,q_2),1]$ (we choose $2\cdot\max(q_1,q_2)$ as the lower interval limit because $\beta>2\cdot q_1,2\cdot q_2$ has to be fulfilled, otherwise the fractional moments don't exist). 
\\
Alternatively, we can use the result of (\ref{eq2}) by choosing a very small fraction $q$ close to zero $0$. Then, the estimator for $\theta$ given by
\begin{align}
\widetilde{\theta}_q=\frac{\widehat{m}_q^2}{\widehat{m}_{2q}}
\end{align}
is independent of $\beta$. Afterwards we calculate $\hat{\beta}$ by solving (\ref{eq3}) for $\beta$ with $\widetilde{\theta}_q$ on the right side of the equation.
\\
Next to the choice of $q$ or $q_1$ and $q_2$ the choice of the threshold $u$ is important. We have to choose $u$ high enouqh such that the limiting result of (\ref{eq4}) holds, but the higher we choose $u$, the fewer observations we have for the estimates of the parameters $\beta$ and $\theta$.
\\
The above written are the theoretical results, now we want to test them by doing a small simulation study in \texttt{R}:
\\
Therefore we choose the \emph{Max-autoregressive (MAR) Processe} with extremal index $\theta \in (0,1)$ for $(J_i)_{i \in \mathbb{N}}$ which is defined by
\begin{align}
J_1&:=X_1/\theta \\
J_i&:=(1-\theta) J_{i-1} \vee X_i \quad \text{for \ } n \ge 2
\end{align}
where $(X_i)_{i \in \mathbb{N}}$ are i.i.d. unit Fr{\'e}chet distributed random variables (Ferro \& Segers, 2003).
The associated waiting time process $(W_i)_{i \in \mathbb{N}}$ is modeled as independent Pareto distributed random variables with c.d.f.
\begin{align}
	F_W(x):=1-\left(\frac{s}{x}\right)^{\beta} \qquad \text{for } 0<\beta<1, \ s>0 \text{ and } x\ge s.
\end{align}
The Pareto distribution with $s=\Gamma(1-\beta)^{-1/\beta}$ is in the domain of attraction of $D$ such that $b(n)=n^{-1/\beta}$ holds in (\ref{eq4}) (Meerschaert \& Sikorskii, 2012).
\end{document}
